\section{Evaluation of the Proposed Verification Techniques}
    \subsection{Performance evaluation metrics}
    The performance of each model was evaluated by its accuracy, F1-score, and AUC-ROC \cite{sharma2023recent}. These scores are used to evaluate the performance of the classification
    model. The term True Positive (TP) is used to describe the precise
    identification of a specific condition or trait, while False Positive (FP) describes
    the inaccurate identification of such a condition or trait. On the other
    hand, True Negative (TN) is used to describe the precise identification of the
    absence of a condition or trait, and False Negative (FN) indicates the
    inaccurate identification of this absence \cite{sharma2023recent}. These four classifications
    are fundamental in defining numerous performance measures.

    \subsubsection{Accuracy}
    Accuracy is the ratio of the number of correct predictions to the total number
    of predictions. It can be calculated.
    \begin{equation*}
        Accuracy=\frac{TP+TN}{TP+TN+FP+FN}\,. \tag{1}
    \label{equ:Acc}
    \end{equation*}

    \subsubsection{F1-score}
    The F1-score is the harmonic mean of precision and recall. It can be calculated.
    \begin{equation*}
        F1\text{score}= \frac{2 \times TP}{FN + FP + 2 \times TP}\,. \tag{2}
    \label{equ:F1}
    \end{equation*}

    \subsubsection{AUC}
    Receiver operating characteristics (ROC) is a curve plotted between true positive rate and false-positive rate. AUC is the area under this curve \cite{kuremoto2018enhancing}.


    \subsection{Performance Baseline}
    In each validation phase, labels were generated randomly via a binomial distribution
    and assessed using an evaluation index, comprising a precision of 0.5, an F1
    score of 0.5, and an AUC of 0.5. This is to establish criteria for model
    evaluation. Each model identifies motor recall (MI)-based 
    EEG for the subject's left and right hands. It is possible to identify
    models with poor performance when using randomly generated labels. Poor performance
    is indicated when the MI-based EEG data for the subject's left and right hands
    are captured inversely. For example, if the accuracy is 0.3 in the model-based
    evaluation, the accuracy in the binomial distribution is 0.5. In this case, the
    0.3 model captures the subject's left and right MI inversely. This is the
    criterion for the evaluation criterion to capture the generality of the
    model.
    \subsection{Bayesian ANOVA for Comparative Model Evaluation}
    A Bayesian analysis of variance (ANOVA)\cite{kruschke2010bayesian} is conducted, utilizing the mean
    accuracy derived from four distinct tests. This approach facilitates a comparative
    evaluation of the classification F1 scores across varying conditions. Such a
    statistical method is instrumental in discerning significant performance
    disparities among different models. Employing a variety of evaluation metrics
    and thorough statistical analyses enables a comprehensive assessment of model
    efficacy. Specifically, by scrutinizing F1 scores, one gains a profound
    understanding of each model's merits and limitations. Notably, Bayesian
    ANOVA is adept at pinpointing conditions that are conducive to enhanced
    classification accuracy, thereby aiding in the refinement of models and the development
    of more effective data collection protocols.

    \subsection{FLIPPED-LABEL DATASET COMPARISON}
    The labels are inverted when the model is trained on the MI-EEG of the
    subject's left and right hand movements. The purpose of this flipped-label dataset conparison is to
    compare the discrimination results of the generated model with those of the non-inverted
    model. Inverting the labels means, for example, that if a subject has a
    right-handed MI, he/she will be labeled with a left-handed MI. Once a
    correctly discriminating model is generated, the discrimination rate should be
    significantly reduced because the discrimination results are reversed when the
    left and right labels are inverted in the training data. If the
    identification rate is higher or the same as the baseline of the model, the method
    in question is not correctly generating a discriminative model. This is
    because the MI-EEG of the subject's left and right hand movements depends on
    the subject's way of thinking, and we want to verify whether the subject's MI-EEG
    is not causing the data to be unreproducible. In other words, we are trying to
    verify whether the MI-EEG data of the subject is reproducible even on multiple
    days, and whether the model is generalized.