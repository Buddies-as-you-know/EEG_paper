\section{RELATED WORK}
\label{sec:relatedwork}
    \subsection{MI-EEG APPLICATIONS and ROBOTICS}
    \label{subsec:MI-EEG APPLICATIONS and ROBOTICS}
        \subsubsection*{Importance of Real-time Responsiveness and Reproducibility in MI-EEG}
        Efficacy of MI-based EEG  in practical applications hinges on its real-time responsiveness and reproducibility. Real-time operation is critical for immediate and accurate execution of user intentions, while reproducibility ensures consistent performance across different sessions and users. These attributes are paramount in applications where user safety and operational efficiency are at stake \cite{singh2021comprehensive}.

        \subsubsection*{Application in Unmanned Aerial Vehicles (Drones)}
        \label{subsubsec:Application in Unmanned Aerial Vehicles (Drones)}
        Integration of MI-EEG into drone technology will enable people who cannot move their limbs to operate them. This advancement enables real-time operation using brain signals, which is essential in dynamic and unpredictable environments. The repeatability of these brain signal interpretations is crucial for consistent and safe drone navigation. In this way, MI-EEG contributes to both innovation and operational proficiency in drone technology\cite{dumitrescu2021using}.

        \subsubsection*{Use in Motorized Wheelchairs}
        \label{subsubsec:Use in Motorized Wheelchairs}
        In motorized wheelchairs, MI-EEG facilitates realtime movement control, transforming the lives of individuals with mobility impairments \cite{palumbo2021motor,wang2024mi}. The reproducibility of signal interpretation is essential for ensuring consistent and safe wheelchair operation. This technology's realtime responsiveness and reliability significantly enhance user autonomy and mobility.

        \subsubsection*{Application in Humanoid Robot Navigation}
        \label{subsubsec:Application in Humanoid Robot Navigation}
        Humanoid robots equipped with MI-EEG systems offer realtime control through thought, which is essential in navigating complex and hazardous environments\cite{aljalal2019robot}. The reproducibility of these systems is vital for ensuring precise and safe robot maneuvering. Therefore, MI-EEG's reliability and realtime operation extend the functionality and utility of humanoid robots in critical applications.

        \subsubsection*{Functioning of Artificial Limbs}
        In artificial limbs, MI-EEG technology enables real-time movement mimicry and signal interpretation \cite{al2018eeg}. The consistency of these interpretations is crucial for user safety and seamless limb operation. Thus, the reproducibility and real-time responsiveness of MI-EEG are key factors in improving the quality of life for amputees.

    The integration of MI-EEG technology in various applications emphasizes its potential to enhance operational efficiency and quality of life. The importance of real-time responsiveness and reproducibility in these applications underlines the technology's capability to ensure user safety and consistent performance.
    \subsection{MI EEG DATASET}
    Table \ref{tab: Comparison of existing data sets and the data sets used in this study}. show that ma2022\cite{ma2022large} has substantive advantages over other major MI-EEG datasets, in terms of scale and duration. With 25 subjects, ma2022\cite{ma2022large} has over double the next largest cohort of 14 in BNCI2015\_001\cite{faller2012autocalibration} and 54 in Lee2019\_MI\cite{lee2019eeg}. Moreover, it includes multi-day data spanning 5 days, while most datasets recorded single sessions. In terms of duration, are BNCI2014\_004\cite{leeb2007brain}, BNCI2015\_004\cite{scherer2015individually}, and Zhou2016\cite{zhou2016fully} with 2--3 days. However, these data sets have 9, 9, and 4 participants, respectively, and the 25 participants in ma2022 are larger. Additionally, ma2022 uniquely uses consumer-grade BCI devices, making it more accessible and usable for real-world applications. By contrast, nearly all other datasets relied on complex medical-grade equipment. Thus, the table highlights ma2022 as a leader, in terms of participant size, longitudinal span, and practicality through its innovative consumer-grade BCI focus over multiple days.

    The ma2022 dataset has three distinguishing characteristics: adoption of consumer-grade BCI devices, multi-day EEG recordings, and a substantial subject cohort. With 25 participants across 5 days, it has the most extensive longitudinal multi-day data crucial for accounting for EEG signal variability. It uses affordable consumer BCI systems rather than intricate, high-end medical equipment.

    By contrast, the existing datasets employ medical devices and single-day recordings with smaller samples. Its scale and duration uniquely distinguish ma2022 to enable richer insights into EEG patterns over time. Additionally, the pragmatic consumer-grade BCI facilitates applications like drone control requiring minimal intrusive gear.

    The exceptionally large participant pool analyzed longitudinally qualifies ma2022 as an inclusive, pragmatic basis for MI-EEG investigation with advantages over current datasets constrained to medical devices and isolated recordings. It promises deeper revelations into EEG behavior through sizable, multi-day data grounded in easily accessible consumer BCIs usable for real-world problems.
    \subsection{MULTIPLE DAY EEG VALIDATION}
    In the evolving field of BCI research, our study introduces innovative methods to enhance model adaptability and generalization using multi-day MI-based EEG data. Unlike the method by Amjad Abu-Rmileh\cite{kaya2023identifying}, which employs Linear Discriminant Analysis(LDA) Classifier for 18 subjects over four days, our approach involves a more extensive participant base of 25 subjects across the same number of days. This broader subject range is crucial for enhancing the reliability and generalizability of our findings. While Abu-Rmileh's study focuses on EEG power spectrum feature extraction, our method does not utilize feature extraction, distinguishing it in methodology. The objective of Abu-Rmileh's research is to improve BCI training effectiveness, whereas our study concentrates on reproducibility verification.

    In contrast to Esra Kaya's study, which involves an in-depth analysis of a single subject over 20 days using an Ensemble Subspace Discriminant Classifier~\cite{abu2019co}, our research utilizes advanced neural network models. Kaya's study emphasizes optimal channel finding through extensive feature extraction including time, frequency and spatiotemporal and nonlinear features. However, our study diverges by not employing traditional feature extraction techniques, focusing instead on leveraging the inherent capabilities of convolutional networks, transformers, and graph networks. This divergence in approach is a significant differentiator, contributing to our research's uniqueness in the field.

    Comparing our methods with the ma2022 study, which uses a variety of algorithms like Common Spatial Patterns (CSP), Filter Bank Common Spatial Pattern (FBCSP), and adaptive transfer learning for 25 subjects over five days~\cite{ma2022large}, our study similarly involves 25 subjects within the same time frame. However, ma2022's focus on improving cross-session classification in BCI contrasts with our objective of reproducibility verification. Both studies share a commitment to open-access datasets, which is a positive trend toward transparency and collaborative research in the field. However, our research stands out by not relying on feature extraction algorithms, a commonality in ma2022's methodology, and instead opts for a more holistic approach using advanced neural networks.

    In summary, our research introduces a novel approach to BCI studies, emphasizing a larger participant base and advanced neural network models without traditional feature extraction. This methodology sets our research apart from existing studies, contributing uniquely to the field's understanding of model adaptability and generalization in the context of multi-day MI-EEG data.