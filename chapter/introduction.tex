\section{INTRODUCTION}
\label{sec:introduction}
Brain-computer interface (BCI) technology represents a groundbreaking approach, deciphering the brain's motor imagery (MI) to issue commands without necessitating physical movement\cite{benzy2020motor}. This technology is particularly promising as it activates the same neural pathways when imagining a movement like those used during actual movement execution, offering an alternative modality of action for individuals with mobility loss\cite{guillot2009brain}. For example, it opens doors to alternative locomotive methods, such as prosthetic limbs and robotic arms\cite{gupta2020prototype,korik2019decoding,cho2021neurograsp}. Anticipated to reduce fatigue, BCI technology heralds a new paradigm in assistive living\cite{lotze2006motor,mulder2007motor}. Research in MI-EEG is pivotal in advancing medical and assistive technologies, with its progress expected to further catalyze technological innovations.


MI represents a critical cognitive process in both neuroscience and sports psychology\cite{bovend2012practical,cryder2023guided}. It involves the mental simulation of a physical action without its actual execution. This process activates the brain regions associated with motor control and planning\cite{henschke2023engaging,munzert2009cognitive} Such activation mirrors the neural patterns observed during the physical performance of the action.

\vspace{3mm}

The significance of MI lies in its application across various fields. In sports, athletes use it to enhance their performance by mentally rehearsing specific skills\cite{cederstrom2021effect,app12199753}. In rehabilitation, it aids individuals with motor impairments in regaining movement capabilities. Furthermore, MI serves as a tool for understanding the neural mechanisms underlying motor control and learning\cite{tong2017motor, choy2023virtual}.

\vspace{3mm}
The utilization of MI-based EEG classification in a variety of applications,
such as unmanned aerial vehicle (drone) operation, motorized wheelchairs,
navigation systems in humanoid robots, and the functioning of artificial limbs,
necessitates real-time identification that is crucial for the operational efficacy
of these technologies\cite{zhang2018converting,chaudhary2020flexible,huang2019eeg,dumitrescu2021using}. Particularly in the case of drones and electric vehicles,
combination of heightened situational awareness and adept operation is
imperative\cite{liu2020parallel,freitas2019real}. The fidelity of MI-based EEG recognition, requiring both high precision and swift responsiveness, plays a pivotal role due to its direct implications on user safety and the functional proficiency of the technology\cite{liu2017identification}. Thus, enhancement of the reproducibility and reliability of MI-EEG technology is of paramount importance in its development for practical applications.

\vspace{3mm}

This research endeavors to authenticate the reproducibility of motor imagery
(MI)-based electroencephalography (EEG) datasets and the universality of the
discrimination algorithm utilized\cite{zhang2018converting,chaudhary2020flexible,huang2019eeg,dumitrescu2021using}. Traditional EEG investigations have
predominantly concentrated on elevating data precision and optimizing
electrode placement, exemplified by endeavors to transcend the state-of-the-art
(SOTA) in discrimination accuracy in BCI competitions etc\cite{du2023recognition,deng2021advanced,xu2019deep,suemitsu2023effects, sharma2023recent}. Nonetheless, within practical environments, the emphasis shifts to the steadfastness of data
reproducibility and the all-encompassing applicability of the model\cite{nakamura2017ear}. For instance,
MI-based EEG outputs must remain consistent for identical
commands executed by the same subject, irrespective of variations in temporal
and environmental contexts. Similarly, the precision in identifying MI-based
EEG patterns ought to maintain uniformity across differing scenarios. This
study not only scrutinizes the reproducibility and universal applicability
of MI-based EEGs but also aims to significantly enhance the interpretative
breadth and practical deployment of EEG data.


\vspace{3mm}

This investigation delves into the consistency of EEG data reproducibility across various days and among different subjects. The process
of data gathering over extended periods could potentially compromise the uniformity
of the data, as result of factors such as electrode displacement and
intrinsic physiological variances. It is of paramount importance to evaluate
the reproducibility to ascertain EEG's dependability in pragmatic scenarios,
especially considering the impracticality of employing consumer-grade BCIs or the elimination of electrooculograms (EOG) in real-world
applications. The primary objective of this assessment is to ascertain the steadfastness
of MI-based EEG data and the universal applicability of the recognition
models employed. Adopting this methodology could substantially enhance both
the utility and dependability of EEG data for broader applications.


\vspace{3mm}
This investigation scrutinizes the efficacy of extant discrimination models
when applied to novel environments, disparate subjects, and over extended
durations. Predominant models often exhibit heightened accuracy solely in specific,
controlled conditions, exemplified by the results seen in the BCI competition
(as seen in Table \ref{tab: Comparison of existing data sets and the data sets used in this study}). The crux of this study is to validate the universality
of these models by evaluating their adaptability to new, diverse datasets.
Such validation is crucial for confirming the practical effectiveness of EEG-based
systems in real-world scenarios. Consequently, this research offers
innovative directions in both the analytical and practical application of
EEG data. In contrast to traditional EEG research, which is primarily focused on
idealized experimental conditions, this study pioneers in exploring the reproducibility
of EEG data and the universal applicability of models within more realistic settings.
This novel approach is anticipated to significantly bolster the utility and
reliability of EEG technology in actual applications. Ultimately, this study
carves a new path in EEG research, heralding an era of more grounded and practical
applications. Our main contributions are as follows:
\begin{itemize}
    \vspace{1mm}
    \item We investigated EEG consistency over time using a multi-day consumer-grade MI-EEG dataset including 25 subjects.
    \vspace{3mm}
    \item Through a comparison of EEG validation techniques (TDV, DDV, TV, SDTDV), we showed that using a portion of the test day data for model adaptation can yield higher accuracy than using only a pre-trained model. .
    \vspace{3mm}
    \item We confirmed the daily reproducibility of MI-EEG and verified the effectiveness of a generalized identification algorithm that can effectively classify individual EEG patterns.
    \vspace{3mm}
    \item We have pioneered a new approach to evaluating the effectiveness of MI-EEG systems in the real world, making it possible to translate this technology into assistive applications.
    \vspace{3mm}
    \item We made the source code public to improve the reproducibility and transparency of our research.
\end{itemize}