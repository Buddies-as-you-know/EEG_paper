\section{Conclusion}
This research aimed to validate the reproducibility of MI-based EEG datasets and the universality of the discrimination algorithms used. The extensive multi-day MI-EEG dataset with 25 participants enabled the analysis of EEG signal consistency over time. Four EEG verification techniques were compared through experiments, namely, TDV, DDV, TV, and SDTDV. Results demonstrated that incorporating a small portion of test day data for model adaptation (as in DDV) yielded higher accuracy than solely relying on pre-training data (TDV). This highlights the potential of adaptive, real time learning models that account for intra-subject variability in EEG patterns. Meanwhile, the competitive performance of TV indicates that in certain applications, test day training alone may suffice over pre-training approaches. The similarity in outcomes between generalized (TDV) and personalized (SDTDV) training suggests that while individual differences exist in EEG data, generalized models can prove effective for MI classification tasks. Overall, the study verifies the reproducibility of MI-EEG signals over multiple days and the viability of using generalized discrimination algorithms. This research pioneers a pragmatic approach to evaluating the real-world effectiveness of MI-EEG systems beyond laboratory conditions. The methodology and findings pave the way for more adaptive, subject-specific, instantly calibrated EEG interfaces. By demonstrating robust reproducibility and model generalizability, this study marks an important step toward transitioning MI-EEG technology into practical assistive and therapeutic applications. An area for future work is investigating reproducibility over more extended durations spanning months or years.