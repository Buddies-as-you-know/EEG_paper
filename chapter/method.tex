

\section{METHOD}
\subsection{DATASET DESCRIPTION AND SUBJECT}
This study used the extensive and publicly available MI-BCI EEG database, which
consists of five sessions of EEG recordings from 25 participants(13 males
and 12 females; age range, 20–24 years) over 2–3 days. Each session
comprised 100 MI grasping trials with the left and right hands, lasting approximately
7.5 s. Human data collection was approved by the Shanghai Second
Rehabilitation Hospital Ethics Committee (approval No.: ECSHSRH 2018-0101)
under the Declaration of Helsinki. Participants were instructed to imagine
a grasping motion with their left and right hands, guided by visual and
auditory cues. Dynamic MI tasks were recorded using 32 EEG electrodes with an
impedance level of less than $20 k\Omega$ and a sampling frequency of 250 Hz
\cite{ma2022large}. The data preprocessing steps include removing bad segments based on
amplitude thresholding and visual inspection, band-pass filtering with a
0.5--40 Hz finite impulse response filter, and baseline removal.
\subsection{BASE MODEL}
This study employed an end-to-end learning method for MI-based EEG signal
classification. This method was chosen because it increases the systems robustness
and can potentially eliminate outliers and other errors affected by the analyst.
As classifiers, we employed EEG-Inception \cite{zhang2021eeg}, EEGNet \cite{lawhern2018eegnet}, and Schirrmeister
et al.s \cite{schirrmeister2017deep} deep convolutional network (DevConv) models, which are capable
of real-time discrimination with a limited number of electrodes. These models
can be also efficiently trained with a smaller number of channels. As a
baseline for the discriminations obtained with these learning methods, we also
showed the results from assigning random labels to the binary discriminations
for the input.

\subsubsection{EEG-Inception}
The EEG-Inception model is a convolutional neural network architecture that consists
of multiple inception modules, which are designed to capture multi-scale and
multi-level features from EEG signals. The inception module is composed of
parallel convolutional layers with different filter sizes and pooling
operations, followed by concatenation of their outputs. The EEG-Inception
model also includes batch normalization, dropout, and global average pooling
layers to improve the model's generalization and reduce overfitting. The model
is trained using the cross-entropy loss function and the Adam optimizer. The
proposed data augmentation method includes random cropping, time-shifting,
and amplitude scaling of EEG signals to increase the diversity and size of
the training dataset. The EEG-Inception model achieves state-of-the-art
performance on two benchmark BCI datasets for MI classification.

\subsubsection{EEGNet}
The EEGNet architecture is a compact convolutional neural network (CNN)
designed specifically for EEG-based BCI paradigms. It consists of two main components:
a temporal convolutional block and a depth-wise separable convolutional
block. The temporal convolutional block applies a one-dimensional
convolutional filter to the input EEG signal along with the temporal axis, followed
by batch normalization and nonlinearity. The depth-wise separable
convolutional block applies a depth-wise convolutional filter to the output
of the temporal block, followed by a pointwise convolutional filter, batch
normalization, and a nonlinearity. The output of the depth-wise separable
block is then passed through a global average pooling layer and a softmax
layer for classification. EEGNet is designed to be a general-purpose architecture
that can be applied to different EEG-based BCI paradigms and has demonstrated high accuracy while remaining compact.
\subsubsection{DevConv}
The ConvNets used in this study vary in the number of convolutional layers and
other design choices, and can extract local, low-level features from the raw
input and then increasingly more global and high-level features in deeper layers.
The feature extraction process involves convolutions and nonlinearities, and
the higher-level features are represented as compositions of lower-level features.
The training of the ConvNets can involve trial-wise and cropped training
strategies, and the EEG input can be represented in different ways depending
on the architecture. Overall, ConvNets have been highly successful in many application
areas, such as computer vision and speech recognition, often outperforming previous
state-of-the-art methods.

\subsubsection{DENOISING}
We used raw data with noise removed because preprocessing of EEG signals is important to improve signal-to-noise ratio and classification accuracy. To remove the noise, bandpass filtering was performed from 0.5 to 40 Hz using a finite impulse response (FIR) filter. This removed the bad segments\cite{ma2022large}.

\subsubsection{ROW DATA}
We validated these models using raw EEG data. The reasons for using raw data are as follows:

\begin{enumerate}
  \item \textbf{Improved Accuracy} \\
    Using raw EEG data, we can achieve high accuracy in classification tasks. For example, a study using a simple CNN classification method for MI-EEG through electrode pair signals achieved a high accuracy level of up to 98.61\% for specific electrode pairs\cite{lun2020simplified}.\\
  
  \item \textbf{Elimination of Preprocessing Steps} \\
    Raw EEG data can be directly input into systems like CNN, eliminating the need for prior feature extraction or preprocessing. This simplifies the data processing pipeline and reduces computational complexity\cite{tibrewal2022classification}.\\
  
  \item \textbf{Improved Handling of High-Dimensional Data} \\
    Deep learning models like CNNs are suitable for raw EEG data as they can process complex, nonlinear, and high-dimensional data\cite{tibrewal2022classification}.\\
  
  \item \textbf{Facilitating End-to-End Learning} \\
    Using raw EEG data promotes end-to-end learning, which can enhance the performance of BCI systems\cite{tibrewal2022classification}.\\
  
  \item \textbf{Possibility for Noise Analysis} \\
    Raw EEG data include all recorded trials and channels, including those with noise. This is beneficial for designing signal processing algorithms to detect, reject, or remove noise within EEG signals\cite{dreyer2023large}.\\
  
  \item \textbf{Improved Generalizability} \\
    The utilization of raw EEG data to design cross-user BCI machine learning algorithms improves the generalizability of the models\cite{dreyer2023large}.\\
  
  \item \textbf{Understanding Signal Characteristics} \\
    Using raw EEG data allows for the study of how EEG signal characteristics vary across different user profiles and MI tasks\cite{dreyer2023large}.\\
\end{enumerate}
We used raw data for these reasons.