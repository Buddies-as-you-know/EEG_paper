\section{COMPARISON OF THE PROPOSED VERIFICATION TECHNIQUES}
  
    This study aimed to compare and contrast the performance of four different EEG
    verification techniques, namely, tri-dataset verification (TDV), dual-dataset
    verification (DDV), temporal verification (TV), and subject dependent tri-dataset
    verification (SDTDV)(at see in Fig \ref{fig: COMPARISON OF THE PROPOSED VERIFICATION TECHNIQUES if day 5 is a test}). In the following sections, we delve into each technique,
    delineate its unique methodology, and evaluate its efficacy. Additionally, as
    mentioned earlier, the Random function was used to assign binary labels entirely
    randomly, without using machine learning.
    \subsection{TDV}
    The TDV approach involves training a model on 4 days of a 5-day dataset for all
    subjects. Pseudo-online validation is performed on the remaining 1-day
    dataset as the test day. The primary goal of this process is to evaluate the
    accuracy of the models classification performance on the test day by training
    it on EEG data that are not from the test day. The EEG signals to be
    discriminated against should be collected during the actual use of the
    equipment, and a pre-trained model should ensure that the EEG signals can be
    accurately classified in real time.

    \subsection{DDV}
    The DDV approach has the same 4-day pre-training period as TDV. However, with
    DDV, the first 20\% of the test-day data are used for training. The
    validation item envisioned by DDV is to adapt the pre-trained model to the
    subject by performing additional training on the subjects data on the test
    day, considering variations in headset impedance, environment, and the subjects
    condition. If the validation results show high accuracy, it demonstrates the
    importance of both prior EEG data acquisition and test-day EEG data for model
    learning during the testing phase for real-time EEG identification. In
    addition, for actual use, a few minutes of pre-training is required when
    subjects use the EEG of left- and right-hand movements to generate commands.

    \subsection{TV}
    The TV approach does not use pre-training data for class identification but
    trains on the first 20\% of the EEG data on the test day for each subject and
    evaluates the remaining 80\%
    of the dataset. Whereas TDV and DDV require pre-training models for class identification,
    TV is a case of its negation, indicating that if the accuracy of TVs model is
    high, it does not require pre-training like TDV and DDV, and only EEG data on
    test day is sufficient for model training. This result is significant
    because it simplifies the data collection process and reduces the time and
    resources required for pre-training. On the other hand, it suggests that
    generalizing such models using EEG data is very difficult.

    \subsection{SDTDV}
    The SDTDV approach validates 4 days of pre-training data and 1 day of test
    data for each subject. Whereas TDV and DDV combine data from 25 subjects to
    build a pre-training model, SDTDV uses only each subjects historical data
    to construct a prior learning model. This study aims to
    evaluate the impact of individual differences and measure whether the models
    accuracy improves by training each subject separately. By comparing TDV and
    SDTDV, it is possible to examine whether the EEG output of the subject is identical
    to that of others. Moreover, this comparison can provide valuable insights into
    the effectiveness of the subject specific training for EEG-based identification.